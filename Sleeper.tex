\documentclass [18pt] {article}

\usepackage{setspace}
\usepackage{graphicx}

\begin{document}
	\title {Sleeper}
	\author{Filippo Colli - Pietro Andreoni}
	\date{25.07.2016}
	\maketitle
	\begin{abstract}
	Sleeper is a program that can figure out your sleep cycle analyzing the breath's frequency. It works completely in a non-invasive way; in fact it takes acoustic
	signal from a microphone, and after a low-pass filter, it can isolate and recognize your breaths pattern.
	This is a great way to take information of your sleep cycle, because every sleep-phase is characterized by a well-defined number of breath per minute.
	After a brief introduction about sleep physiology and phases, in this report we will describe software, hardware and pseudo-code that we used in order to
	complete the project.
	\end{abstract}
	\pagebreak
	\section*{Physiology of sleep}
	Sleep is a physiological condition of all humans that typically recurs for several hours every night, in which the nervous system is inactive, the eyes closed, the
	postural muscles relaxed, and consciousness practically suspended, required to maintain health and proper function.
	Despite a sleeping person may seem physically and mentally completely inactive, studies have proven that during sleep time occurs a complex brain
	activity. According to the different characteristics of brain waves during different moments of sleep, it can be divided into two macro-areas:
	\begin{itemize}
	\item REM sleep: the REM sleep is characterized by a mixture of the encephalic state excitement and muscular immobility. The most known peculiarity of REM 
	sleep is the rapid eyes movement from which the phase takes his name. It is the phase in which the most part of dreams occurs and in a physiological subject 
	returns cyclically 3 or 4 times per sleep. REM sleep become more frequent and last longer into the final parts of the sleep. In this stadio vegetative functions are
	less controlled. Breathing patterns in particular are characterized by an higher frequency (24-36 breath per minute) than NREM sleep or awake condition and by 
	a great variability.
	\item NREM sleep: the NREM sleep is characterized by low frequency and high amplitude brain waves, little or no eyes movement and not paralyzed muscular
	activity. It can be divided into several further stages, but all of them are characterized by a very regular breath pattern, between 3 or 4 breath per minute.
	\item awakeness or light-sleep:  during the night, especially after REM phases, the subject can experience brief moments of awakeness, in which breath
	frequency is around 12-18 breath per minute.
	\end{itemize}
	\section*{Algorithm}
	The algorithm is composed by two main stages. 
	\subsection*{}
	The first one, that we will call sampling phase, takes the signal from the microphone with a sampling rate of
	10Hz. Then it is filtered by a low-pass filter with a cut-frequency of 2Hz and a gain of 2.0. We decided this configuration for several reasons: we chose an high 
	sampling rate, despite the fact that the frequency information contained in breath event is fully described by lower harmonics frequencies, because it would 
	easily allow in future to implement further functions requiring higher frequency contents. On the other hand the cut frequency of 2.0 Hz causes a loss of 
	information about the shape of the respiratory event; in particular, we are not able to distinguish if the subject is snoring and, in general, to infer about differences 
	and characteristics of a single breath event. Thou, this choice of filter simplified a lot the analysis of breathing frequency: in fact, as the filtered signal can 
	be considered as a sine-like wave, the research of respiratory events became a simple maximum analysis. The stage we described so far repeats cyclically every 
	60 seconds, and the only information we decided to maintain after the end of every cycle is the number of breaths. This way we were able to save a lot of static
	memory if compared to an algorithm that saves the whole sampled signal.
	\subsection*{} 
	The second stage is not cyclically repeated, it activates after the end of the sleep (which is reported from subject changing a switch) and it analyzes the
	information received from stage one, i.e. breathing frequencies of the whole sleep time.
	In particular, we analyzed the different sleep stages occurred during the night, showing as output an hypnogram, which contains the information about sleep 
	stages. Then we analyzed and selected informations about REM phases, the most significant of sleep phases in order to understand the health of sleep, 
	showing duration and occurring time for each one.
	
	\begin{figure}
    	 \centering
    	 \includegraphics[scale=.6] {../../Downloads/sleeper_img.png}
     	\caption{Pseudo-code of Sleeper}      
	\end{figure}
\end{document}

	


